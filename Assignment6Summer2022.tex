% Options for packages loaded elsewhere
\PassOptionsToPackage{unicode}{hyperref}
\PassOptionsToPackage{hyphens}{url}
%
\documentclass[
]{article}
\usepackage{amsmath,amssymb}
\usepackage{lmodern}
\usepackage{iftex}
\ifPDFTeX
  \usepackage[T1]{fontenc}
  \usepackage[utf8]{inputenc}
  \usepackage{textcomp} % provide euro and other symbols
\else % if luatex or xetex
  \usepackage{unicode-math}
  \defaultfontfeatures{Scale=MatchLowercase}
  \defaultfontfeatures[\rmfamily]{Ligatures=TeX,Scale=1}
\fi
% Use upquote if available, for straight quotes in verbatim environments
\IfFileExists{upquote.sty}{\usepackage{upquote}}{}
\IfFileExists{microtype.sty}{% use microtype if available
  \usepackage[]{microtype}
  \UseMicrotypeSet[protrusion]{basicmath} % disable protrusion for tt fonts
}{}
\makeatletter
\@ifundefined{KOMAClassName}{% if non-KOMA class
  \IfFileExists{parskip.sty}{%
    \usepackage{parskip}
  }{% else
    \setlength{\parindent}{0pt}
    \setlength{\parskip}{6pt plus 2pt minus 1pt}}
}{% if KOMA class
  \KOMAoptions{parskip=half}}
\makeatother
\usepackage{xcolor}
\IfFileExists{xurl.sty}{\usepackage{xurl}}{} % add URL line breaks if available
\IfFileExists{bookmark.sty}{\usepackage{bookmark}}{\usepackage{hyperref}}
\hypersetup{
  pdftitle={STAT 2450 Assignment 6 {[}40 pts{]}},
  pdfauthor={B00841761},
  hidelinks,
  pdfcreator={LaTeX via pandoc}}
\urlstyle{same} % disable monospaced font for URLs
\usepackage[margin=1in]{geometry}
\usepackage{color}
\usepackage{fancyvrb}
\newcommand{\VerbBar}{|}
\newcommand{\VERB}{\Verb[commandchars=\\\{\}]}
\DefineVerbatimEnvironment{Highlighting}{Verbatim}{commandchars=\\\{\}}
% Add ',fontsize=\small' for more characters per line
\usepackage{framed}
\definecolor{shadecolor}{RGB}{248,248,248}
\newenvironment{Shaded}{\begin{snugshade}}{\end{snugshade}}
\newcommand{\AlertTok}[1]{\textcolor[rgb]{0.94,0.16,0.16}{#1}}
\newcommand{\AnnotationTok}[1]{\textcolor[rgb]{0.56,0.35,0.01}{\textbf{\textit{#1}}}}
\newcommand{\AttributeTok}[1]{\textcolor[rgb]{0.77,0.63,0.00}{#1}}
\newcommand{\BaseNTok}[1]{\textcolor[rgb]{0.00,0.00,0.81}{#1}}
\newcommand{\BuiltInTok}[1]{#1}
\newcommand{\CharTok}[1]{\textcolor[rgb]{0.31,0.60,0.02}{#1}}
\newcommand{\CommentTok}[1]{\textcolor[rgb]{0.56,0.35,0.01}{\textit{#1}}}
\newcommand{\CommentVarTok}[1]{\textcolor[rgb]{0.56,0.35,0.01}{\textbf{\textit{#1}}}}
\newcommand{\ConstantTok}[1]{\textcolor[rgb]{0.00,0.00,0.00}{#1}}
\newcommand{\ControlFlowTok}[1]{\textcolor[rgb]{0.13,0.29,0.53}{\textbf{#1}}}
\newcommand{\DataTypeTok}[1]{\textcolor[rgb]{0.13,0.29,0.53}{#1}}
\newcommand{\DecValTok}[1]{\textcolor[rgb]{0.00,0.00,0.81}{#1}}
\newcommand{\DocumentationTok}[1]{\textcolor[rgb]{0.56,0.35,0.01}{\textbf{\textit{#1}}}}
\newcommand{\ErrorTok}[1]{\textcolor[rgb]{0.64,0.00,0.00}{\textbf{#1}}}
\newcommand{\ExtensionTok}[1]{#1}
\newcommand{\FloatTok}[1]{\textcolor[rgb]{0.00,0.00,0.81}{#1}}
\newcommand{\FunctionTok}[1]{\textcolor[rgb]{0.00,0.00,0.00}{#1}}
\newcommand{\ImportTok}[1]{#1}
\newcommand{\InformationTok}[1]{\textcolor[rgb]{0.56,0.35,0.01}{\textbf{\textit{#1}}}}
\newcommand{\KeywordTok}[1]{\textcolor[rgb]{0.13,0.29,0.53}{\textbf{#1}}}
\newcommand{\NormalTok}[1]{#1}
\newcommand{\OperatorTok}[1]{\textcolor[rgb]{0.81,0.36,0.00}{\textbf{#1}}}
\newcommand{\OtherTok}[1]{\textcolor[rgb]{0.56,0.35,0.01}{#1}}
\newcommand{\PreprocessorTok}[1]{\textcolor[rgb]{0.56,0.35,0.01}{\textit{#1}}}
\newcommand{\RegionMarkerTok}[1]{#1}
\newcommand{\SpecialCharTok}[1]{\textcolor[rgb]{0.00,0.00,0.00}{#1}}
\newcommand{\SpecialStringTok}[1]{\textcolor[rgb]{0.31,0.60,0.02}{#1}}
\newcommand{\StringTok}[1]{\textcolor[rgb]{0.31,0.60,0.02}{#1}}
\newcommand{\VariableTok}[1]{\textcolor[rgb]{0.00,0.00,0.00}{#1}}
\newcommand{\VerbatimStringTok}[1]{\textcolor[rgb]{0.31,0.60,0.02}{#1}}
\newcommand{\WarningTok}[1]{\textcolor[rgb]{0.56,0.35,0.01}{\textbf{\textit{#1}}}}
\usepackage{graphicx}
\makeatletter
\def\maxwidth{\ifdim\Gin@nat@width>\linewidth\linewidth\else\Gin@nat@width\fi}
\def\maxheight{\ifdim\Gin@nat@height>\textheight\textheight\else\Gin@nat@height\fi}
\makeatother
% Scale images if necessary, so that they will not overflow the page
% margins by default, and it is still possible to overwrite the defaults
% using explicit options in \includegraphics[width, height, ...]{}
\setkeys{Gin}{width=\maxwidth,height=\maxheight,keepaspectratio}
% Set default figure placement to htbp
\makeatletter
\def\fps@figure{htbp}
\makeatother
\setlength{\emergencystretch}{3em} % prevent overfull lines
\providecommand{\tightlist}{%
  \setlength{\itemsep}{0pt}\setlength{\parskip}{0pt}}
\setcounter{secnumdepth}{-\maxdimen} % remove section numbering
\ifLuaTeX
  \usepackage{selnolig}  % disable illegal ligatures
\fi

\title{STAT 2450 Assignment 6 {[}40 pts{]}}
\author{B00841761}
\date{}

\begin{document}
\maketitle

\hypertarget{problem-1-classification-with-cart-30-pts}{%
\subsection{Problem 1: Classification with CART {[}30
pts{]}}\label{problem-1-classification-with-cart-30-pts}}

\hypertarget{preparing-the-data-0-pts}{%
\subsubsection{Preparing the data {[}0
pts{]}}\label{preparing-the-data-0-pts}}

Download the data6a.csv dataframe from BS and use read.csv to read the
data.

Call `d' the resulting dataframe.

\begin{Shaded}
\begin{Highlighting}[]
\NormalTok{d }\OtherTok{\textless{}{-}} \FunctionTok{read.csv}\NormalTok{(}\AttributeTok{file=}\StringTok{\textquotesingle{}data6a.csv\textquotesingle{}}\NormalTok{)}
\end{Highlighting}
\end{Shaded}

Which columns are categorical predictors?

Use as.factor() to redefine each categorical variable as a factor.

Apply the function `str' to d to check that the categorical columns are
now factors

\begin{Shaded}
\begin{Highlighting}[]
\CommentTok{\#d$colname = as.factor(d$colname)}
\end{Highlighting}
\end{Shaded}

\begin{Shaded}
\begin{Highlighting}[]
\FunctionTok{rm}\NormalTok{(d)}
\NormalTok{d }\OtherTok{\textless{}{-}} \FunctionTok{read.csv}\NormalTok{(}\AttributeTok{file=}\StringTok{\textquotesingle{}data6a.csv\textquotesingle{}}\NormalTok{)}
\FunctionTok{str}\NormalTok{(d)}
\end{Highlighting}
\end{Shaded}

\begin{verbatim}
## 'data.frame':    394 obs. of  11 variables:
##  $ x1 : int  104 116 127 116 133 145 139 107 113 141 ...
##  $ x2 : int  102 83 45 26 97 30 24 67 100 63 ...
##  $ x3 : int  13 15 19 6 0 0 0 12 16 0 ...
##  $ x4 : int  123 170 459 434 70 67 358 430 353 168 ...
##  $ x5 : int  110 144 129 115 117 104 185 92 79 135 ...
##  $ x6 : chr  "Good" "Bad" "Medium" "Medium" ...
##  $ x7 : int  35 71 57 25 32 55 79 35 68 44 ...
##  $ x8 : int  16 11 11 17 16 17 15 12 11 12 ...
##  $ x9 : chr  "Yes" "Yes" "No" "Yes" ...
##  $ x10: chr  "Yes" "Yes" "Yes" "Yes" ...
##  $ x11: chr  "Yes " "No" "No" "No" ...
\end{verbatim}

\begin{Shaded}
\begin{Highlighting}[]
\NormalTok{d}\SpecialCharTok{$}\NormalTok{x6}\OtherTok{=}\FunctionTok{as.factor}\NormalTok{(d}\SpecialCharTok{$}\NormalTok{x6)}
\NormalTok{d}\SpecialCharTok{$}\NormalTok{x9}\OtherTok{=}\FunctionTok{as.factor}\NormalTok{(d}\SpecialCharTok{$}\NormalTok{x9)}
\NormalTok{d}\SpecialCharTok{$}\NormalTok{x10}\OtherTok{=}\FunctionTok{as.factor}\NormalTok{(d}\SpecialCharTok{$}\NormalTok{x10)}
\NormalTok{d}\SpecialCharTok{$}\NormalTok{x11}\OtherTok{=}\FunctionTok{as.factor}\NormalTok{(d}\SpecialCharTok{$}\NormalTok{x11)}
\FunctionTok{str}\NormalTok{(d)}
\end{Highlighting}
\end{Shaded}

\begin{verbatim}
## 'data.frame':    394 obs. of  11 variables:
##  $ x1 : int  104 116 127 116 133 145 139 107 113 141 ...
##  $ x2 : int  102 83 45 26 97 30 24 67 100 63 ...
##  $ x3 : int  13 15 19 6 0 0 0 12 16 0 ...
##  $ x4 : int  123 170 459 434 70 67 358 430 353 168 ...
##  $ x5 : int  110 144 129 115 117 104 185 92 79 135 ...
##  $ x6 : Factor w/ 3 levels "Bad","Good","Medium": 2 1 3 3 3 3 3 3 1 1 ...
##  $ x7 : int  35 71 57 25 32 55 79 35 68 44 ...
##  $ x8 : int  16 11 11 17 16 17 15 12 11 12 ...
##  $ x9 : Factor w/ 2 levels "No","Yes": 2 2 1 2 2 2 1 1 2 2 ...
##  $ x10: Factor w/ 2 levels "No","Yes": 2 2 2 2 1 1 1 2 2 2 ...
##  $ x11: Factor w/ 2 levels "No","Yes ": 2 1 1 1 2 2 1 2 2 1 ...
\end{verbatim}

Use the following syntax to rename the column x11 as y.

\begin{Shaded}
\begin{Highlighting}[]
\FunctionTok{names}\NormalTok{(d)[}\FunctionTok{names}\NormalTok{(d) }\SpecialCharTok{==} \StringTok{\textquotesingle{}old.var.name\textquotesingle{}}\NormalTok{] }\OtherTok{\textless{}{-}} \StringTok{\textquotesingle{}new.var.name\textquotesingle{}}
\end{Highlighting}
\end{Shaded}

Run the function `str' on d to check that the renaming has worked
correctly.

\begin{Shaded}
\begin{Highlighting}[]
\FunctionTok{names}\NormalTok{(d)[}\FunctionTok{names}\NormalTok{(d) }\SpecialCharTok{==} \StringTok{\textquotesingle{}x11\textquotesingle{}}\NormalTok{] }\OtherTok{\textless{}{-}} \StringTok{\textquotesingle{}y\textquotesingle{}}
\FunctionTok{str}\NormalTok{(d)}
\end{Highlighting}
\end{Shaded}

\begin{verbatim}
## 'data.frame':    394 obs. of  11 variables:
##  $ x1 : int  104 116 127 116 133 145 139 107 113 141 ...
##  $ x2 : int  102 83 45 26 97 30 24 67 100 63 ...
##  $ x3 : int  13 15 19 6 0 0 0 12 16 0 ...
##  $ x4 : int  123 170 459 434 70 67 358 430 353 168 ...
##  $ x5 : int  110 144 129 115 117 104 185 92 79 135 ...
##  $ x6 : Factor w/ 3 levels "Bad","Good","Medium": 2 1 3 3 3 3 3 3 1 1 ...
##  $ x7 : int  35 71 57 25 32 55 79 35 68 44 ...
##  $ x8 : int  16 11 11 17 16 17 15 12 11 12 ...
##  $ x9 : Factor w/ 2 levels "No","Yes": 2 2 1 2 2 2 1 1 2 2 ...
##  $ x10: Factor w/ 2 levels "No","Yes": 2 2 2 2 1 1 1 2 2 2 ...
##  $ y  : Factor w/ 2 levels "No","Yes ": 2 1 1 1 2 2 1 2 2 1 ...
\end{verbatim}

\hypertarget{goal}{%
\section{Goal}\label{goal}}

The goal of this problem is to model the binary response (column `y'),
as a function of all other predictors (columns `x1' to `x10') available
in dataframe `d'. You will use the `tree' library to fit a CART model to
the data.

\hypertarget{training-set}{%
\section{Training set}\label{training-set}}

Write a R code to define a training set (name it `dtrain') and a testing
set (name it `dtest'.

Use `ytrain' as the name of the response for the training set, and
`ytest' as the name of the response for the testing set. `ytest' will be
useful for the calculation of testing errors.

For exact replication, we will use this random seed:

\begin{Shaded}
\begin{Highlighting}[]
\CommentTok{\#set.seed(666)}
\end{Highlighting}
\end{Shaded}

Use the `sample' function (with no replacement) to define a random index
vector named `index' of 300 records from `d':

\begin{Shaded}
\begin{Highlighting}[]
\CommentTok{\#index= sample(1:nrow(d), 300, replace=FALSE)}
\end{Highlighting}
\end{Shaded}

Use `index' to define your training set (named `dtrain'), and its
complement to define the testing set (named `dtest').

Define a vector `ytrain' as the `y' column of `dtrain' and a vector
`ytest' as the `y' column of `dtest':

\begin{Shaded}
\begin{Highlighting}[]
\CommentTok{\# dtrain=d[index,]}
\CommentTok{\# dtest =d[{-}index,]}
\CommentTok{\# ytrain=d$y[index]}
\CommentTok{\# ytest =d$y[{-}index]}
\end{Highlighting}
\end{Shaded}

define dtrain, dtest, ytrain, ytest by sampling 300

\begin{Shaded}
\begin{Highlighting}[]
\FunctionTok{set.seed}\NormalTok{(}\DecValTok{666}\NormalTok{)}
\FunctionTok{nrow}\NormalTok{(d)}
\end{Highlighting}
\end{Shaded}

\begin{verbatim}
## [1] 394
\end{verbatim}

\begin{Shaded}
\begin{Highlighting}[]
\NormalTok{ns}\OtherTok{=}\DecValTok{300}
\NormalTok{index}\OtherTok{=}\FunctionTok{sample}\NormalTok{(}\DecValTok{1}\SpecialCharTok{:}\FunctionTok{nrow}\NormalTok{(d),ns,}\AttributeTok{replace=}\NormalTok{F)}
\NormalTok{dtrain}\OtherTok{=}\NormalTok{d[index,]}
\NormalTok{dtest }\OtherTok{=}\NormalTok{d[}\SpecialCharTok{{-}}\NormalTok{index,]}
\NormalTok{ytrain}\OtherTok{=}\NormalTok{d}\SpecialCharTok{$}\NormalTok{y[index]}
\NormalTok{ytest }\OtherTok{=}\NormalTok{d}\SpecialCharTok{$}\NormalTok{y[}\SpecialCharTok{{-}}\NormalTok{index]}
\end{Highlighting}
\end{Shaded}

\hypertarget{model-fitting-4-pts}{%
\subsubsection{Model fitting {[}4 pts{]}}\label{model-fitting-4-pts}}

\begin{enumerate}
\def\labelenumi{\arabic{enumi}.}
\item
  Fit a tree model (call the resulting object `dtraintree') to the
  training data, with `y' as the response and all the other variables as
  predictors.
\item
  Use the `summary()' function to produce summary statistics about the
  tree (save the summary into an object called `Sd').
\item
  What is the training error rate? Ans: 0.06667
\item
  How many terminal nodes does the tree have? Ans: 24
\end{enumerate}

\begin{Shaded}
\begin{Highlighting}[]
\FunctionTok{library}\NormalTok{(tree)}
\CommentTok{\# fit the CART model}
\NormalTok{dtraintree}\OtherTok{=}\FunctionTok{tree}\NormalTok{(y}\SpecialCharTok{\textasciitilde{}}\NormalTok{.,}\AttributeTok{data=}\NormalTok{dtrain)}
\NormalTok{Sd}\OtherTok{=}\FunctionTok{summary}\NormalTok{(dtraintree)}
\CommentTok{\# Print the summary object:}
\NormalTok{Sd}
\end{Highlighting}
\end{Shaded}

\begin{verbatim}
## 
## Classification tree:
## tree(formula = y ~ ., data = dtrain)
## Variables actually used in tree construction:
## [1] "x6" "x5" "x2" "x1" "x4" "x7" "x3"
## Number of terminal nodes:  24 
## Residual mean deviance:  0.3581 = 98.84 / 276 
## Misclassification error rate: 0.06667 = 20 / 300
\end{verbatim}

\hypertarget{tree-interpretation-3-pts}{%
\subsubsection{Tree interpretation {[}3
pts{]}}\label{tree-interpretation-3-pts}}

Type in the name of the tree object in order to get a detailed text
output.

Pick one of the terminal nodes (leaves), and interpret the information
displayed for this node (i.e.~explain the meaning of each term).

Ans: 31) x5 \textgreater{} 142.5 8 10.590 No ( 0.62500 0.37500 ) *

\begin{enumerate}
\def\labelenumi{\arabic{enumi})}
\setcounter{enumi}{30}
\item
\begin{verbatim}
            --> an unique number for each node of the tree
\end{verbatim}

  x5 \textgreater{} 142.5 --\textgreater{} this is the split, equation
  used to branch at the node 10.590 --\textgreater{} associated deviance
  with the branch No --\textgreater{} the predicted value at the node (
  0.62500 0.37500 ) --\textgreater{} the proportion probability of
  values in that branch that are absent and present
\end{enumerate}

\begin{itemize}
\item
\begin{verbatim}
              --> used to demote a terminal node
\end{verbatim}
\end{itemize}

\begin{Shaded}
\begin{Highlighting}[]
\NormalTok{dtraintree}
\end{Highlighting}
\end{Shaded}

\begin{verbatim}
## node), split, n, deviance, yval, (yprob)
##       * denotes terminal node
## 
##   1) root 300 402.100 No ( 0.60667 0.39333 )  
##     2) x6: Bad,Medium 243 295.300 No ( 0.70370 0.29630 )  
##       4) x5 < 92.5 40  50.450 Yes  ( 0.32500 0.67500 )  
##         8) x2 < 57 10  12.220 No ( 0.70000 0.30000 )  
##          16) x1 < 110.5 5   0.000 No ( 1.00000 0.00000 ) *
##          17) x1 > 110.5 5   6.730 Yes  ( 0.40000 0.60000 ) *
##         9) x2 > 57 30  30.020 Yes  ( 0.20000 0.80000 )  
##          18) x4 < 221 14  18.250 Yes  ( 0.35714 0.64286 )  
##            36) x5 < 85.5 9   6.279 Yes  ( 0.11111 0.88889 ) *
##            37) x5 > 85.5 5   5.004 No ( 0.80000 0.20000 ) *
##          19) x4 > 221 16   7.481 Yes  ( 0.06250 0.93750 ) *
##       5) x5 > 92.5 203 214.800 No ( 0.77833 0.22167 )  
##        10) x1 < 124.5 96  59.740 No ( 0.90625 0.09375 )  
##          20) x2 < 62.5 45   0.000 No ( 1.00000 0.00000 ) *
##          21) x2 > 62.5 51  47.530 No ( 0.82353 0.17647 )  
##            42) x7 < 66 30  36.650 No ( 0.70000 0.30000 )  
##              84) x5 < 121 21  28.680 No ( 0.57143 0.42857 )  
##               168) x1 < 110 9   6.279 No ( 0.88889 0.11111 ) *
##               169) x1 > 110 12  15.280 Yes  ( 0.33333 0.66667 )  
##                 338) x3 < 6.5 6   7.638 No ( 0.66667 0.33333 ) *
##                 339) x3 > 6.5 6   0.000 Yes  ( 0.00000 1.00000 ) *
##              85) x5 > 121 9   0.000 No ( 1.00000 0.00000 ) *
##            43) x7 > 66 21   0.000 No ( 1.00000 0.00000 ) *
##        11) x1 > 124.5 107 136.700 No ( 0.66355 0.33645 )  
##          22) x5 < 122.5 44  57.680 Yes  ( 0.36364 0.63636 )  
##            44) x1 < 137.5 33  45.720 Yes  ( 0.48485 0.51515 )  
##              88) x6: Bad 6   0.000 No ( 1.00000 0.00000 ) *
##              89) x6: Medium 27  35.590 Yes  ( 0.37037 0.62963 )  
##               178) x5 < 109.5 10   0.000 Yes  ( 0.00000 1.00000 ) *
##               179) x5 > 109.5 17  23.030 No ( 0.58824 0.41176 )  
##                 358) x3 < 5 10  10.010 No ( 0.80000 0.20000 ) *
##                 359) x3 > 5 7   8.376 Yes  ( 0.28571 0.71429 ) *
##            45) x1 > 137.5 11   0.000 Yes  ( 0.00000 1.00000 ) *
##          23) x5 > 122.5 63  47.960 No ( 0.87302 0.12698 )  
##            46) x7 < 54.5 39  39.580 No ( 0.79487 0.20513 )  
##              92) x2 < 63.5 24   8.314 No ( 0.95833 0.04167 ) *
##              93) x2 > 63.5 15  20.730 No ( 0.53333 0.46667 )  
##               186) x4 < 267 6   5.407 Yes  ( 0.16667 0.83333 ) *
##               187) x4 > 267 9   9.535 No ( 0.77778 0.22222 ) *
##            47) x7 > 54.5 24   0.000 No ( 1.00000 0.00000 ) *
##     3) x6: Good 57  55.920 Yes  ( 0.19298 0.80702 )  
##       6) x5 < 121 30   0.000 Yes  ( 0.00000 1.00000 ) *
##       7) x5 > 121 27  36.500 Yes  ( 0.40741 0.59259 )  
##        14) x2 < 35.5 5   0.000 No ( 1.00000 0.00000 ) *
##        15) x2 > 35.5 22  25.780 Yes  ( 0.27273 0.72727 )  
##          30) x5 < 142.5 14   7.205 Yes  ( 0.07143 0.92857 ) *
##          31) x5 > 142.5 8  10.590 No ( 0.62500 0.37500 ) *
\end{verbatim}

\hypertarget{plot-of-tree-3-pts}{%
\subsubsection{Plot of tree {[}3 pts{]}}\label{plot-of-tree-3-pts}}

Create a plot of the tree, apply the text labels, and write a sentence
to explain the meaning of the first node split.

Ans: The first node split implies the best predictor (independent
variables).

Be careful with levels of factors if the node splits a categorical
variable.

\begin{Shaded}
\begin{Highlighting}[]
\FunctionTok{plot}\NormalTok{(dtraintree)}
\FunctionTok{text}\NormalTok{(dtraintree)}
\end{Highlighting}
\end{Shaded}

\includegraphics{Assignment6Summer2022_files/figure-latex/unnamed-chunk-12-1.pdf}

\hypertarget{testing-errors-4-pts}{%
\subsubsection{Testing errors {[}4 pts{]}}\label{testing-errors-4-pts}}

Use the `predict' function to predict the response on the testing data,
and produce a confusion matrix comparing the test labels to the
predicted test labels.

Compute the value of the testing error rate (define a variable called
`testrate' and calculate it from the confusion table). Apply the `round'
function to print or report `testrate' with 4 digits after the decimal
point.

\begin{Shaded}
\begin{Highlighting}[]
\NormalTok{dprune}\OtherTok{=}\FunctionTok{prune.misclass}\NormalTok{(dtraintree,}\AttributeTok{best=}\DecValTok{5}\NormalTok{)}

\NormalTok{pred.test}\OtherTok{=}\FunctionTok{predict}\NormalTok{(dprune,dtest,}\AttributeTok{type=}\StringTok{"class"}\NormalTok{)}
\NormalTok{ctable}\OtherTok{=}\FunctionTok{table}\NormalTok{(ytest,pred.test)}
\FunctionTok{print}\NormalTok{(ctable)}
\end{Highlighting}
\end{Shaded}

\begin{verbatim}
##       pred.test
## ytest  No Yes 
##   No   33   17
##   Yes  11   33
\end{verbatim}

\begin{Shaded}
\begin{Highlighting}[]
\NormalTok{testrate}\OtherTok{=}\NormalTok{ (}\FunctionTok{sum}\NormalTok{(ctable)}\SpecialCharTok{{-}}\FunctionTok{sum}\NormalTok{(}\FunctionTok{diag}\NormalTok{(ctable)))}\SpecialCharTok{/}\FunctionTok{sum}\NormalTok{(ctable)}
\end{Highlighting}
\end{Shaded}

\hypertarget{cross-validation-2-pts}{%
\subsubsection{Cross-validation {[}2
pts{]}}\label{cross-validation-2-pts}}

Apply the cv.tree() function to the training set. Call the resulting
object `dcv'. We will use this object in the next questions, especially
to determine the optimal tree size.

Choose the appropriate option for pruning your tree in your call to
cv.tree (we want to measure lack of fit or deviance by the
misclassification error rate). Print the `dcv' object.

\begin{Shaded}
\begin{Highlighting}[]
\FunctionTok{cv.tree}\NormalTok{(dtraintree)}
\end{Highlighting}
\end{Shaded}

\begin{verbatim}
## $size
##  [1] 24 23 22 20 19 17 16 15 14 12 11 10  8  7  6  5  3  2  1
## 
## $dev
##  [1] 596.0753 587.1902 566.9817 562.8218 562.8218 546.0319 531.5219 531.5219
##  [9] 511.1186 469.3916 465.2928 459.4699 459.7643 445.4942 441.1940 403.5816
## [17] 385.5137 366.4929 405.3990
## 
## $k
##  [1]      -Inf  4.650987  5.487169  5.629905  5.786253  7.382496  7.969718
##  [8]  7.991981  8.205051  9.457856 10.716740 10.880066 11.341298 11.964990
## [15] 12.204874 19.420484 24.702759 30.110549 50.871335
## 
## $method
## [1] "deviance"
## 
## attr(,"class")
## [1] "prune"         "tree.sequence"
\end{verbatim}

\begin{Shaded}
\begin{Highlighting}[]
\NormalTok{dcv}\OtherTok{=}\FunctionTok{cv.tree}\NormalTok{(dtraintree, }\AttributeTok{FUN=}\NormalTok{prune.misclass)}
\NormalTok{dcv}
\end{Highlighting}
\end{Shaded}

\begin{verbatim}
## $size
##  [1] 24 23 18 15 13 12 10  6  5  3  2  1
## 
## $dev
##  [1]  67  73  73  73  69  70  73  64  71  79  88 118
## 
## $k
##  [1]      -Inf  1.000000  1.200000  1.333333  1.500000  2.000000  2.500000
##  [8]  3.000000  4.000000  6.000000 14.000000 35.000000
## 
## $method
## [1] "misclass"
## 
## attr(,"class")
## [1] "prune"         "tree.sequence"
\end{verbatim}

\hypertarget{cv-plot-3-pts}{%
\subsubsection{CV plot {[}3 pts{]}}\label{cv-plot-3-pts}}

Use `dcv' to produce a plot with tree size on the x-axis and
cross-validated classification error rate on the y-axis.

Use the argument type=`b' in your plot and label your axis (x-label:
tree size, y-label: error-rate).

\begin{Shaded}
\begin{Highlighting}[]
\FunctionTok{plot}\NormalTok{(dcv}\SpecialCharTok{$}\NormalTok{size,dcv}\SpecialCharTok{$}\NormalTok{dev,}\AttributeTok{type=}\StringTok{\textquotesingle{}b\textquotesingle{}}\NormalTok{, }\AttributeTok{xlab=}\StringTok{"tree size"}\NormalTok{, }\AttributeTok{ylab =} \StringTok{"error{-}rate"}\NormalTok{)}
\end{Highlighting}
\end{Shaded}

\includegraphics{Assignment6Summer2022_files/figure-latex/unnamed-chunk-15-1.pdf}

\hypertarget{optimal-complexity-3-pts}{%
\subsubsection{Optimal complexity {[}3
pts{]}}\label{optimal-complexity-3-pts}}

Which tree size corresponds to the lowest cross-validated classification
error rate?

\begin{Shaded}
\begin{Highlighting}[]
\CommentTok{\# compute bestsize from dcv:}
\NormalTok{bestsize}\OtherTok{=}\NormalTok{dcv}\SpecialCharTok{$}\NormalTok{size[dcv}\SpecialCharTok{$}\NormalTok{dev}\SpecialCharTok{==}\FunctionTok{min}\NormalTok{(dcv}\SpecialCharTok{$}\NormalTok{dev)]}
\CommentTok{\# and print it:}
\NormalTok{bestsize}
\end{Highlighting}
\end{Shaded}

\begin{verbatim}
## [1] 6
\end{verbatim}

The tree with 6 terminal nodes has the lowest cross-validated
classification error rate.

It might be more logical to choose the first local minimum instead of
the global minimum. What would this tree size be for this choice?

We will assume that the optimal tree size is 5 for the next question.

\hypertarget{pruning-3-pts}{%
\subsubsection{Pruning {[}3 pts{]}}\label{pruning-3-pts}}

Produce a pruned tree named `dprune' corresponding to the optimal tree
size obtained using cross-validation. Here, overwrite this best size by
5 to create a pruned tree with five terminal nodes. Plot the `dprune'
tree and add text labels.

\begin{Shaded}
\begin{Highlighting}[]
\NormalTok{dprune}\OtherTok{=}\FunctionTok{prune.misclass}\NormalTok{(dtraintree,}\AttributeTok{best=}\DecValTok{5}\NormalTok{)}
\FunctionTok{plot}\NormalTok{(dprune)}
\FunctionTok{text}\NormalTok{(dprune)}
\end{Highlighting}
\end{Shaded}

\includegraphics{Assignment6Summer2022_files/figure-latex/unnamed-chunk-17-1.pdf}

\hypertarget{compare-training-errors-3-pts}{%
\subsubsection{Compare training errors {[}3
pts{]}}\label{compare-training-errors-3-pts}}

Compare the training error rates obtained with the pruned tree to those
obtained with the unpruned trees.

In order to do this, use the `predict' function and apply it to either
the `dtraintree' object (unpruned tree model) or to the `dprune' object
(pruned tree model).

Do not forget to specify the argument `newdata' and the argument `type'.

Which error rate is higher?

\begin{Shaded}
\begin{Highlighting}[]
\CommentTok{\# calculate the prediction for the unpruned tree:}
\NormalTok{predtrainu}\OtherTok{=} \FunctionTok{predict}\NormalTok{(dtraintree,}\AttributeTok{data=}\NormalTok{dtrain, }\AttributeTok{type=}\StringTok{"class"}\NormalTok{)}

\CommentTok{\# calculate the prediction for the pruned tree:}
\NormalTok{predtrainp}\OtherTok{=}\FunctionTok{predict}\NormalTok{(dprune,}\AttributeTok{data=}\NormalTok{dtrain, }\AttributeTok{type=}\StringTok{"class"}\NormalTok{)}

\CommentTok{\# compute the confusion matrix of the training error for the unpruned tree model}

\NormalTok{ctableu}\OtherTok{=} \FunctionTok{table}\NormalTok{(dtraintree}\SpecialCharTok{$}\NormalTok{y, predtrainu)}

\CommentTok{\# print a message telling the reader that you will print the}
\CommentTok{\# training error confusion matrix for the unpruned tree:}

\FunctionTok{print}\NormalTok{(}\StringTok{"Training error confusion matrix for unpruned tree: "}\NormalTok{)}
\end{Highlighting}
\end{Shaded}

\begin{verbatim}
## [1] "Training error confusion matrix for unpruned tree: "
\end{verbatim}

\begin{Shaded}
\begin{Highlighting}[]
\CommentTok{\# print the confusion matrix:}

\NormalTok{ctableu}
\end{Highlighting}
\end{Shaded}

\begin{verbatim}
##       predtrainu
##         No Yes 
##   No   174    8
##   Yes   12  106
\end{verbatim}

\begin{Shaded}
\begin{Highlighting}[]
\CommentTok{\# compute the training error rate of the unpruned tree}
\CommentTok{\#trainrateu= (ctableu["No", "No"]+ctableu["No", "Yes"])}
\NormalTok{trainrateu }\OtherTok{=}\NormalTok{ (}\DecValTok{174}\SpecialCharTok{+}\DecValTok{8}\NormalTok{)}\SpecialCharTok{/}\NormalTok{(}\DecValTok{8}\SpecialCharTok{+}\DecValTok{106}\SpecialCharTok{+}\DecValTok{174}\SpecialCharTok{+}\DecValTok{12}\NormalTok{)}
\NormalTok{trainrateu}
\end{Highlighting}
\end{Shaded}

\begin{verbatim}
## [1] 0.6066667
\end{verbatim}

\begin{Shaded}
\begin{Highlighting}[]
\CommentTok{\# print a message telling the reader that you will print the}
\CommentTok{\# training error confusion matrix for the pruned tree:}

\FunctionTok{print}\NormalTok{(}\StringTok{"Training error confusionn matrix for the pruned tree:"}\NormalTok{)}
\end{Highlighting}
\end{Shaded}

\begin{verbatim}
## [1] "Training error confusionn matrix for the pruned tree:"
\end{verbatim}

\begin{Shaded}
\begin{Highlighting}[]
\CommentTok{\# compute the confusion matrix of the training error for the pruned tree model}

\NormalTok{ctablep}\OtherTok{=} \FunctionTok{table}\NormalTok{(dtraintree}\SpecialCharTok{$}\NormalTok{y, predtrainp)}

\CommentTok{\# print it}
\NormalTok{ctablep}
\end{Highlighting}
\end{Shaded}

\begin{verbatim}
##       predtrainp
##         No Yes 
##   No   142   40
##   Yes   17  101
\end{verbatim}

\begin{Shaded}
\begin{Highlighting}[]
\CommentTok{\# calculate the training error rate of the  pruned tree model}

\NormalTok{trainratep}\OtherTok{=}\NormalTok{(}\DecValTok{142}\SpecialCharTok{+}\DecValTok{17}\NormalTok{)}\SpecialCharTok{/}\NormalTok{(}\DecValTok{142}\SpecialCharTok{+}\DecValTok{17}\SpecialCharTok{+}\DecValTok{40}\SpecialCharTok{+}\DecValTok{101}\NormalTok{)}
\NormalTok{trainratep}
\end{Highlighting}
\end{Shaded}

\begin{verbatim}
## [1] 0.53
\end{verbatim}

The training error rates are 0.6066667 for the unpruned tree, and 0.53
for the pruned tree. The training error rate is higher for the
pruned/unpruned? tree.

Ans: The training error rate is higher in the unpruned tree.

\hypertarget{compare-testing-errors-2-pts}{%
\subsubsection{Compare testing errors {[}2
pts{]}}\label{compare-testing-errors-2-pts}}

Compare the testing error rates on obtained with the pruned and with the
unpruned trees.

Which is higher?

This is the same as before but for the records in the testing dataset.

\begin{Shaded}
\begin{Highlighting}[]
\NormalTok{dtesttree}\OtherTok{=}\FunctionTok{tree}\NormalTok{(y}\SpecialCharTok{\textasciitilde{}}\NormalTok{.,}\AttributeTok{data=}\NormalTok{dtest)}
\NormalTok{dtestprune}\OtherTok{=}\FunctionTok{prune.misclass}\NormalTok{(dtesttree,}\AttributeTok{best=}\DecValTok{5}\NormalTok{)}

\NormalTok{predtestu}\OtherTok{=} \FunctionTok{predict}\NormalTok{(dtesttree,}\AttributeTok{data=}\NormalTok{dtest, }\AttributeTok{type=}\StringTok{"class"}\NormalTok{)}
\NormalTok{predtestp}\OtherTok{=} \FunctionTok{predict}\NormalTok{(dtestprune,}\AttributeTok{data=}\NormalTok{dtest, }\AttributeTok{type=}\StringTok{"class"}\NormalTok{)}

\NormalTok{ctableu}\OtherTok{=} \FunctionTok{table}\NormalTok{(dtesttree}\SpecialCharTok{$}\NormalTok{y, predtestu)}

\FunctionTok{print}\NormalTok{(}\StringTok{"Training error confusion matrix for unpruned tree: "}\NormalTok{)}
\end{Highlighting}
\end{Shaded}

\begin{verbatim}
## [1] "Training error confusion matrix for unpruned tree: "
\end{verbatim}

\begin{Shaded}
\begin{Highlighting}[]
\NormalTok{ctableu}
\end{Highlighting}
\end{Shaded}

\begin{verbatim}
##       predtestu
##        No Yes 
##   No   49    1
##   Yes   9   35
\end{verbatim}

\begin{Shaded}
\begin{Highlighting}[]
\NormalTok{testrateu}\OtherTok{=}\NormalTok{ (}\DecValTok{49}\SpecialCharTok{+}\DecValTok{9}\NormalTok{)}\SpecialCharTok{/}\NormalTok{(}\DecValTok{49}\SpecialCharTok{+}\DecValTok{1}\SpecialCharTok{+}\DecValTok{9}\SpecialCharTok{+}\DecValTok{35}\NormalTok{)}
\NormalTok{testrateu}
\end{Highlighting}
\end{Shaded}

\begin{verbatim}
## [1] 0.6170213
\end{verbatim}

\begin{Shaded}
\begin{Highlighting}[]
\NormalTok{ctablep}\OtherTok{=} \FunctionTok{table}\NormalTok{(dtesttree}\SpecialCharTok{$}\NormalTok{y, predtestp)}

\FunctionTok{print}\NormalTok{(}\StringTok{"Training error confusionn matrix for the pruned tree:"}\NormalTok{)}
\end{Highlighting}
\end{Shaded}

\begin{verbatim}
## [1] "Training error confusionn matrix for the pruned tree:"
\end{verbatim}

\begin{Shaded}
\begin{Highlighting}[]
\NormalTok{ctablep}
\end{Highlighting}
\end{Shaded}

\begin{verbatim}
##       predtestp
##        No Yes 
##   No   38   12
##   Yes   5   39
\end{verbatim}

\begin{Shaded}
\begin{Highlighting}[]
\NormalTok{testratep}\OtherTok{=}\NormalTok{ (}\DecValTok{38}\SpecialCharTok{+}\DecValTok{5}\NormalTok{)}\SpecialCharTok{/}\NormalTok{(}\DecValTok{38}\SpecialCharTok{+}\DecValTok{12}\SpecialCharTok{+}\DecValTok{5}\SpecialCharTok{+}\DecValTok{39}\NormalTok{)}
\NormalTok{testratep}
\end{Highlighting}
\end{Shaded}

\begin{verbatim}
## [1] 0.4574468
\end{verbatim}

The test error rates are 0.6170213 for the unpruned tree, and 0.4574468
for the pruned tree. The test error rate is higher for the
pruned/unpruned? tree.

Ans: The error rate for the unpruned tree is higher.

\newpage

\hypertarget{gini-index-and-information-gain-total-10-pts}{%
\subsection{Gini index and information gain {[}Total: 10
pts{]}}\label{gini-index-and-information-gain-total-10-pts}}

The method that you need to use in this problem will be discussed during
the live lectures this week (Module 6 - Lecture 2)

\hypertarget{gini-index-function-5-pts}{%
\subsubsection{Gini index function {[}5
pts{]}}\label{gini-index-function-5-pts}}

Write a R function called mygini that takes two binary-valued vectors x
and y (of same length) as arguments, and returns the Gini index of the
split associated with the predictor x.

The binary vectors x and y take only the values 0 and 1 and represent
the predictor column and the response column of a dataframe.

Hint: use the following formula, with a proper calculations of all
probabilities:

\[Gi =q_1*(1-p_{11}^2-p_{01}^2)+q_0*(1-p_{10}^2-p_{00}^2)\]

\(q_1\) is the proportion of records in the node of the split that is
found in the subset (\(x=1\)), and \(q_0=1-q_1\) is the proportion found
in the complementary subset of the split (\(x=0\)).

To calculate the conditional probabilities:

\[ p_{ij}=P(Y=i|X=j)\]

you can use boolean expressions (with the AND operator:\&) and sum of
booleans.

For example I can calculate the probability q1 like this:

\begin{Shaded}
\begin{Highlighting}[]
\NormalTok{   x}\OtherTok{=}\FunctionTok{c}\NormalTok{(}\DecValTok{1}\NormalTok{,}\DecValTok{1}\NormalTok{,}\DecValTok{1}\NormalTok{,}\DecValTok{0}\NormalTok{,}\DecValTok{0}\NormalTok{,}\DecValTok{0}\NormalTok{,}\DecValTok{0}\NormalTok{,}\DecValTok{1}\NormalTok{,}\DecValTok{0}\NormalTok{)}
\NormalTok{   q1}\OtherTok{=}\FunctionTok{sum}\NormalTok{(x}\SpecialCharTok{==}\DecValTok{1}\NormalTok{)}\SpecialCharTok{/}\FunctionTok{length}\NormalTok{(x)}
\end{Highlighting}
\end{Shaded}

Implement your function `mygini' here:

\begin{Shaded}
\begin{Highlighting}[]
\NormalTok{mygini }\OtherTok{=} \ControlFlowTok{function}\NormalTok{(x,y) \{}
\NormalTok{  q1}\OtherTok{=}\FunctionTok{sum}\NormalTok{(x}\SpecialCharTok{==}\DecValTok{1}\NormalTok{)}\SpecialCharTok{/}\FunctionTok{length}\NormalTok{(x)}
\NormalTok{  q0}\OtherTok{=}\DecValTok{1}\SpecialCharTok{{-}}\NormalTok{q1}
\NormalTok{  p11}\OtherTok{=}\FunctionTok{sum}\NormalTok{(x}\SpecialCharTok{==}\DecValTok{1} \SpecialCharTok{\&}\NormalTok{ y}\SpecialCharTok{==}\DecValTok{1}\NormalTok{) }\SpecialCharTok{/}\FunctionTok{sum}\NormalTok{(x}\SpecialCharTok{==}\DecValTok{1}\NormalTok{)}
\NormalTok{  p01}\OtherTok{=}\DecValTok{1}\SpecialCharTok{{-}}\NormalTok{p11}
\NormalTok{  p10}\OtherTok{=}\FunctionTok{sum}\NormalTok{(x}\SpecialCharTok{==}\DecValTok{0} \SpecialCharTok{\&}\NormalTok{ y}\SpecialCharTok{==}\DecValTok{1}\NormalTok{) }\SpecialCharTok{/} \FunctionTok{sum}\NormalTok{(x}\SpecialCharTok{==}\DecValTok{0}\NormalTok{)}
\NormalTok{  p00}\OtherTok{=}\DecValTok{1}\SpecialCharTok{{-}}\NormalTok{p10}
\NormalTok{  gini}\OtherTok{=}\NormalTok{q1}\SpecialCharTok{*}\NormalTok{(}\DecValTok{1}\SpecialCharTok{{-}}\NormalTok{(p11)}\SpecialCharTok{\^{}}\DecValTok{2}\SpecialCharTok{{-}}\NormalTok{(p01)}\SpecialCharTok{\^{}}\DecValTok{2}\NormalTok{)}\SpecialCharTok{+}\NormalTok{q0}\SpecialCharTok{*}\NormalTok{(}\DecValTok{1}\SpecialCharTok{{-}}\NormalTok{(p10)}\SpecialCharTok{\^{}}\DecValTok{2}\SpecialCharTok{{-}}\NormalTok{(p00)}\SpecialCharTok{\^{}}\DecValTok{2}\NormalTok{)}
  \FunctionTok{return}\NormalTok{ (gini)}
\NormalTok{\}}
\end{Highlighting}
\end{Shaded}

Check your function by calling it on the following data. You should get
the indicated response.

\begin{Shaded}
\begin{Highlighting}[]
\NormalTok{x}\OtherTok{=}\FunctionTok{c}\NormalTok{(}\DecValTok{1}\NormalTok{,}\DecValTok{1}\NormalTok{,}\DecValTok{0}\NormalTok{,}\DecValTok{1}\NormalTok{,}\DecValTok{1}\NormalTok{,}\DecValTok{1}\NormalTok{,}\DecValTok{0}\NormalTok{,}\DecValTok{0}\NormalTok{,}\DecValTok{1}\NormalTok{)}
\NormalTok{y }\OtherTok{=}\FunctionTok{c}\NormalTok{(}\DecValTok{1}\NormalTok{,}\DecValTok{1}\NormalTok{,}\DecValTok{0}\NormalTok{,}\DecValTok{1}\NormalTok{,}\DecValTok{0}\NormalTok{,}\DecValTok{0}\NormalTok{,}\DecValTok{0}\NormalTok{,}\DecValTok{1}\NormalTok{,}\DecValTok{0}\NormalTok{)}
\CommentTok{\# call your function mygini (uncomment the next line)}
\FunctionTok{mygini}\NormalTok{(x,y) }\CommentTok{\# this should return the value: 0.4814815}
\end{Highlighting}
\end{Shaded}

\begin{verbatim}
## [1] 0.4814815
\end{verbatim}

\hypertarget{information-gain-function-5-pts}{%
\subsubsection{Information gain function {[}5
pts{]}}\label{information-gain-function-5-pts}}

Write a function named `myinfogain' that calculates the information gain
due to the split of a binary predictor x.

Hint: The information gain can be calculated as:

\[\Delta=H-q_1 H_{x=1}-q_0 H_{x=0}\] where \(H_y\) is the entropy of
\(y\), and \(H_{x=a}\) is the conditional entropy of \(y\) given that we
are in the subset \(x=a\).

The detailed calculations of each term of this formula are as follows (P
means probability of the event inside the bracket):

\[ H=-(P(y=1)\log P(y=1) +P(y=0)\log P(y=0))   \\ 
H_{x=1} = -(P(y=1|x=1)\log P(y=1|x=1) +P(y=0|x=1)\log P(y=0|x=1)) \\ H_{x=0}  = -(P(y=1|x=0)\log P(y=1|x=0) +P(y=0|x=0)\log P(y=0|x=0)) \\ q_1 = P(x=1)  \\ q_0=P(x=0)\]

You can see most of the terms that need to be computed are of the form:
\[ h(p)=   p\times \log(p)\]

To calculate each of these term, you will use the function xlogx:

\begin{Shaded}
\begin{Highlighting}[]
\NormalTok{xlogx }\OtherTok{\textless{}{-}} \ControlFlowTok{function}\NormalTok{(x)\{}
\ControlFlowTok{if}\NormalTok{(x}\SpecialCharTok{==}\DecValTok{0}\NormalTok{)r}\OtherTok{=}\DecValTok{0}
\ControlFlowTok{if}\NormalTok{(x}\SpecialCharTok{\textgreater{}}\DecValTok{0}\NormalTok{)r}\OtherTok{=}\NormalTok{x}\SpecialCharTok{*}\FunctionTok{log2}\NormalTok{(x)}
\FunctionTok{return}\NormalTok{(r)}
\NormalTok{\}}

\CommentTok{\# for example this will calculate 0.3 * log 0.3}
\NormalTok{p}\OtherTok{=}\FloatTok{0.3}
\FunctionTok{xlogx}\NormalTok{(p)}
\end{Highlighting}
\end{Shaded}

\begin{verbatim}
## [1] -0.5210897
\end{verbatim}

\begin{Shaded}
\begin{Highlighting}[]
\CommentTok{\# note that log(0) is {-} infinity but 0 * log (0) is well defined and should be 0}

\NormalTok{p}\OtherTok{=}\DecValTok{0}
\FunctionTok{print}\NormalTok{(}\StringTok{"This will fail : "}\NormalTok{)}
\end{Highlighting}
\end{Shaded}

\begin{verbatim}
## [1] "This will fail : "
\end{verbatim}

\begin{Shaded}
\begin{Highlighting}[]
\NormalTok{p}\SpecialCharTok{*}\FunctionTok{log2}\NormalTok{(p)  }\CommentTok{\# will fail to produce 0}
\end{Highlighting}
\end{Shaded}

\begin{verbatim}
## [1] NaN
\end{verbatim}

\begin{Shaded}
\begin{Highlighting}[]
\FunctionTok{print}\NormalTok{(}\StringTok{"This is correct : 0*log(0)=0 :  "}\NormalTok{)}
\end{Highlighting}
\end{Shaded}

\begin{verbatim}
## [1] "This is correct : 0*log(0)=0 :  "
\end{verbatim}

\begin{Shaded}
\begin{Highlighting}[]
\FunctionTok{xlogx}\NormalTok{(p)   }\CommentTok{\# will calculate the correct value: 0}
\end{Highlighting}
\end{Shaded}

\begin{verbatim}
## [1] 0
\end{verbatim}

Implement the function `myinfogain' here:

\begin{Shaded}
\begin{Highlighting}[]
\NormalTok{myinfogain }\OtherTok{\textless{}{-}} \ControlFlowTok{function}\NormalTok{(x,y)\{}

\NormalTok{  p1}\OtherTok{=}\FunctionTok{sum}\NormalTok{(y}\SpecialCharTok{==}\DecValTok{1}\NormalTok{)}\SpecialCharTok{/}\FunctionTok{length}\NormalTok{(y)}
\NormalTok{  p0}\OtherTok{=}\DecValTok{1}\SpecialCharTok{{-}}\NormalTok{p1}
  
\NormalTok{  hy}\OtherTok{=}\SpecialCharTok{{-}}\NormalTok{(}\FunctionTok{xlogx}\NormalTok{(p0)}\SpecialCharTok{+}\FunctionTok{xlogx}\NormalTok{(p1))}
  
\NormalTok{  p1}\OtherTok{=}\FunctionTok{sum}\NormalTok{(y}\SpecialCharTok{==}\DecValTok{1} \SpecialCharTok{\&}\NormalTok{ x}\SpecialCharTok{==}\DecValTok{1}\NormalTok{)}\SpecialCharTok{/}\FunctionTok{sum}\NormalTok{( x}\SpecialCharTok{==}\DecValTok{1}\NormalTok{)}
\NormalTok{  p0}\OtherTok{=}\DecValTok{1}\SpecialCharTok{{-}}\NormalTok{p1}
\NormalTok{  hx1}\OtherTok{=}\SpecialCharTok{{-}}\NormalTok{(}\FunctionTok{xlogx}\NormalTok{(p0)}\SpecialCharTok{+}\FunctionTok{xlogx}\NormalTok{(p1))}
  
\NormalTok{  p1}\OtherTok{=}\FunctionTok{sum}\NormalTok{(y}\SpecialCharTok{==}\DecValTok{1} \SpecialCharTok{\&}\NormalTok{ x}\SpecialCharTok{==}\DecValTok{0}\NormalTok{)}\SpecialCharTok{/}\FunctionTok{sum}\NormalTok{( x}\SpecialCharTok{==}\DecValTok{0}\NormalTok{)}
\NormalTok{  p0}\OtherTok{=}\DecValTok{1}\SpecialCharTok{{-}}\NormalTok{p1}
\NormalTok{  hx0}\OtherTok{=}\SpecialCharTok{{-}}\NormalTok{(}\FunctionTok{xlogx}\NormalTok{(p0)}\SpecialCharTok{+}\FunctionTok{xlogx}\NormalTok{(p1))}
  
\NormalTok{  q1}\OtherTok{=}\FunctionTok{sum}\NormalTok{(x}\SpecialCharTok{==}\DecValTok{1}\NormalTok{)}\SpecialCharTok{/}\FunctionTok{length}\NormalTok{(x)}
\NormalTok{  q0}\OtherTok{=}\DecValTok{1}\SpecialCharTok{{-}}\NormalTok{q1}
  
  
\NormalTok{  delta}\OtherTok{=}\NormalTok{hy}\SpecialCharTok{{-}}\NormalTok{q1}\SpecialCharTok{*}\NormalTok{hx1}\SpecialCharTok{{-}}\NormalTok{q0}\SpecialCharTok{*}\NormalTok{hx0}
  
  
  \FunctionTok{return}\NormalTok{(delta)}
\NormalTok{\}}
\end{Highlighting}
\end{Shaded}

Check your function by calling it on the following data. You should get
the indicated response.

\begin{Shaded}
\begin{Highlighting}[]
\NormalTok{x}\OtherTok{=}\FunctionTok{c}\NormalTok{(}\DecValTok{1}\NormalTok{, }\DecValTok{1}\NormalTok{, }\DecValTok{1}\NormalTok{, }\DecValTok{1}\NormalTok{, }\DecValTok{0}\NormalTok{, }\DecValTok{0}\NormalTok{, }\DecValTok{0}\NormalTok{, }\DecValTok{1}\NormalTok{, }\DecValTok{0}\NormalTok{, }\DecValTok{0}\NormalTok{)}
\NormalTok{y}\OtherTok{=}\FunctionTok{c}\NormalTok{(}\DecValTok{1}\NormalTok{, }\DecValTok{1}\NormalTok{, }\DecValTok{1}\NormalTok{, }\DecValTok{0}\NormalTok{, }\DecValTok{1}\NormalTok{, }\DecValTok{0}\NormalTok{, }\DecValTok{0}\NormalTok{, }\DecValTok{0}\NormalTok{, }\DecValTok{0}\NormalTok{, }\DecValTok{0}\NormalTok{)}
\FunctionTok{myinfogain}\NormalTok{(x,y) }\CommentTok{\# this should produce the result: 0.1245112}
\end{Highlighting}
\end{Shaded}

\begin{verbatim}
## [1] 0.1245112
\end{verbatim}

\end{document}
